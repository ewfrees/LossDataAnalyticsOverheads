\documentclass{beamer}

\mode<presentation> {
  \usetheme{Madison}
%    \usetheme[left,hideallsubsections,width=1cm]{UWThemeB}
  \usefonttheme[onlymath]{serif}
}

% % % % to make navigation two lines
\usepackage{ragged2e}

\makeatletter
\setbeamertemplate{section in head/foot}{%
	\parbox[c][0.33cm][t]{\dimexpr(\textwidth-1.3cm)/\beamer@sectionmax\relax}{%
		\RaggedRight\fontsize{4}{4}\selectfont\insertsectionhead}}
\setbeamertemplate{footline}[frame number]
\makeatother
% % % %
\usepackage{booktabs, calc, rotating}
\usepackage{scalefnt}
\usepackage[english]{babel}
\usepackage[latin1]{inputenc}
\usepackage{times}
%\usepackage[T1]{fontenc}
\usepackage{alltt}

  \setbeamertemplate{navigation symbols}{}
  \setbeamertemplate{blocks}[rounded][shadow=true]

%%%%%%%%%%%%%%%%%%%%%%%%%%%%%%%%%%%%%%%%%%%%%%%%%%%%%%%%%%%%%%%%%%%%%%%%%%%%%%%%%%%%
%  CHANGE THE TITLE AND INPUT FILE ACCORDING TO THE CHAPTER THAT YOU WISH TO COMPILE
%%%%%%%%%%%%%%%%%%%%%%%%%%%%%%%%%%%%%%%%%%%%%%%%%%%%%%%%%%%%%%%%%%%%%%%%%%%%%%%%%%%%

\title[Portfolio Management]{Portfolio Management including Reinsurance}


\date[Fall 2017]{Fall 2017}
%Personal definitions - Bayes Regression
\def\bsb{\boldsymbol \beta}
\def\bsa{\boldsymbol \alpha}
\def\bsm{\boldsymbol \mu}
\def\bsS{\boldsymbol \Sigma}
\def\bsx{\boldsymbol \xi}

\begin{document}

\frame{\titlepage}

\begin{frame}
  \frametitle{Outline}
    %\tableofcontents[part=1,pausesections]
     \tableofcontents[part=1]
\end{frame}

\part<presentation>{Main Talk}


\begin{frame}%[shrink=2]
\frametitle{Overview: Portfolio Management including Reinsurance}
Define $S$ to be (random) obligations that arise from a collection (portfolio) of insurance contracts
\vspace{0.3cm}
\begin{itemize}
\item We are particularly interested in probabilities of large outcomes and so formalize the notion of a ``heavy-tail'' distribution
\item How much in assets does an insurer need to retain to meet obligations arising from the random $S$? A study of ``risk measures'' helps to address this question
\item As with policyholders, insurers also seek mechanisms in order to spread risks. A company that sells insurance to an insurance company is known as a ``reinsurer''
\end{itemize}
\end{frame}

\section{Tails of Distributions}

\begin{frame}{Tails of Distributions}
\begin{itemize}
\item The \textcolor{blue}{tail} of a distribution (more specifically: the \textcolor{blue}{right tail}) is the portion of the distribution corresponding to large values of the r.v.
\vspace{0.3cm}
\item Understanding large possible loss values is important because they have the greatest impact on the total of losses.
\vspace{0.3cm}
\item R.v.'s that tend to assign higher probabilities to larger values are said to be \textcolor{blue}{heavier tailed}.
\vspace{0.3cm}
\item When choosing models, tail weight can help narrow choices or can confirm a choice for a model.
\end{itemize}
\end{frame}

\begin{frame}{Classification Based on Moments}
\begin{itemize}
\item One way of classifying distributions:
\vspace{0.3cm}
\item[] are all moments finite, or not?
\vspace{0.3cm}
\item The \textcolor{blue}{finiteness} of \textcolor{blue}{all positive moments} indicates a (relatively) light right tail.
\vspace{0.3cm}
\item The \textcolor{blue}{finiteness} of only positive moments \textcolor{blue}{up to a certain value} indicates a heavy right tail.
\end{itemize}
\end{frame}

\begin{frame}{Classification Based on Moments}
\begin{itemize}
\item \textcolor{blue}{KPW Example 3.9}: demonstrate that for the gamma distribution all positive moments are finite but for the Pareto distribution they are not.
\vspace{0.3cm}
\item For the gamma distribution
\begin{eqnarray*}
\mu_k^{'} &=& \int_0^{\infty} x^k \frac{x^{\alpha-1} e^{-x/\theta}}{\Gamma(\alpha) \theta^{\alpha}} dx \\
&=& \int_0^{\infty} (y\theta)^k  \frac{(y\theta)^{\alpha-1} e^{-y}}{\Gamma(\alpha) \theta^{\alpha}} \theta dy \\
&=& \frac{\theta^k}{\Gamma(\alpha)} \Gamma(\alpha+k) < \infty \ \ \ \text{for\ all}\ k>0.
\end{eqnarray*}
\end{itemize}
\end{frame}

\begin{frame}{Classification Based on Moments II}
\begin{itemize}
\item \textcolor{blue}{KPW Example 3.9}: demonstrate that for the gamma distribution all positive moments exist but for the Pareto distribution they do not.
\item For the Pareto distribution
\begin{eqnarray*}
\mu_k^{'} &=& \int_0^{\infty} x^k \frac{\alpha \theta^{\alpha}}{(x+\theta)^{\alpha+1}} dx \\
&=& \int_{\theta}^{\infty} (y-\theta)^k \frac{\alpha \theta^{\alpha}}{y^{\alpha+1}} dy \\
&=& \alpha \cdot \theta^{\alpha} \int_{\theta}^{\infty} \sum_{j=0}^k \left(\begin{array}{c}
                                                                             k \\
                                                                             j
                                                                           \end{array} \right) y^{j-\alpha-1} (-\theta)^{k-j} dy,
\end{eqnarray*}
for integer values of $k$.
\vspace{0.3cm}
\item[] This integral is finite only if $\int_{\theta}^{\infty} y^{j-\alpha-1} dy = \frac{y^{j-\alpha}}{j-\alpha}\big{|}_{\theta}^{\infty}$ is finite.
\item[] Finiteness occurs when $j-\alpha < 0$ for $j=1, \ldots,k$. Or, equivalently, $k< \alpha$.
\end{itemize}
\end{frame}

\begin{frame}{Classification Based on Moments III}
\begin{itemize}
\item Pareto is said to have a heavy tail, and gamma has a light tail.
\end{itemize}
\end{frame}

\begin{frame}{Comparison Based on Limiting Tail Behavior}
\begin{itemize}
\item Consider two distributions with the same mean.
\vspace{0.3cm}
\item If ratio of $S_1(.)$ and $S_2(.)$ diverges to infinity, then distribution 1 has a heavier tail than distribution 2.
\vspace{0.3cm}
\item Thus, we examine
\begin{eqnarray*}
\lim_{x\to \infty} \frac{S_1(x)}{S_2(x)} &=& \lim_{x \to \infty} \frac{S_1^{'}(x)}{S_2^{'}(x)} \\
&=& \lim_{x \to \infty} \frac{-f_1(x)}{-f_2(x)} = \lim_{x\to \infty} \frac{f_1(x)}{f_2(x)}.
\end{eqnarray*}
\end{itemize}
\end{frame}

\begin{frame}{Comparison Based on Limiting Tail Behavior II}
\begin{itemize}
\item \textcolor{blue}{KPW Example 3.10}: demonstrate that Pareto distribution has a heavier tail than the gamma distribution using the limit of the ratio of their density functions.
\vspace{0.3cm}
\item We consider
\begin{eqnarray*}
\lim_{x\to \infty} \frac{f_{\text{Pareto}}(x)}{f_{\text{gamma}}(x)} &=& \lim_{x \to \infty} \frac{\alpha \theta^{\alpha} (x+ \theta)^{-\alpha-1}}{x^{\tau-1} e^{-x/\lambda} \lambda^{-\tau} \Gamma(\tau)^{-1}} \\
&=& c \lim_{x\to \infty} \frac{e^{x/\lambda}}{(x+\theta)^{\alpha+1} x^{\tau-1}} \\
&=& \infty
\end{eqnarray*}
\item[] Exponentials go to infinity faster than polynomials, thus the limit is infinity.
\end{itemize}
\end{frame}

\section{Measures of Risk}

\begin{frame}{Measures of Risk}
\begin{itemize}
\item A \textcolor{blue}{risk measure} is a mapping from the r.v. representing the loss associated with the risks to the real line.
\vspace{0.3cm}
\item A risk measure gives a single number that is intended to quantify the risk.
\vspace{0.3cm}
\begin{itemize}\item For example, the standard deviation is a risk measure.\end{itemize}
\vspace{0.3cm}
\item Notation: $\rho(X)$.
\vspace{0.4cm}
\item We briefly mention:
\vspace{0.3cm}
\begin{itemize}
\item[-] \textcolor{blue}{VaR}: Value at Risk;
\item[-] \textcolor{blue}{TVaR}: Tail Value at Risk.
\end{itemize}
\end{itemize}
\end{frame}

\begin{frame}{Value at Risk}
\begin{itemize}
\item Say $F_X(x)$ represents the cdf of outcomes over a fixed period of time, e.g.\ one year, of a portfolio of risks.
\vspace{0.3cm}
\item We consider positive values of $X$ as losses.
\vspace{0.3cm}
\item \textcolor{blue}{Definition 3.11}: let $X$ denote a loss r.v., then the \textcolor{blue}{Value-at-Risk} of $X$ at the $100p\%$ level, denoted $VaR_p(X)$ or $\pi_p$, is the $100p$ percentile (or quantile) of the distribution of $X$.
\vspace{0.3cm}
\item E.g. for continuous distributions we have
\begin{eqnarray*}
P(X> \pi_p) &=& 1-p.
\end{eqnarray*}
%\item VaR is not subadditive, see the next example.
\end{itemize}
\end{frame}

\begin{frame}{Value at Risk II}
\begin{itemize}
\item VaR has become the standard risk measure used to evaluate exposure to risk.
\vspace{0.3cm}
\item \textcolor{blue}{VaR} is the \textcolor{blue}{amount of capital} required to ensure, with a \textcolor{blue}{high degree of certainty}, that the \textcolor{blue}{enterprise does not become technically insolvent}.
\vspace{0.3cm}
\item Which degree of certainty?
\vspace{0.3cm}
\begin{itemize}
\item[-] 95$\%$?
\vspace{0.3cm}
\item[-] in Solvency II $99.5\%$ (or: ruin probability of 1 in 200).
\end{itemize}
\vspace{0.3cm}
\end{itemize}
\end{frame}



\begin{frame}{Value at Risk III}
\begin{itemize}
\item \textcolor{blue}{VaR is not subadditive}.
\begin{itemize}
\item[] Subadditivity of a risk measure $\rho(.)$ requires
\begin{eqnarray*}
\rho(X+Y) \leq \rho(X)+\rho(Y).
\end{eqnarray*}
\item[] Intuition behind subadditivity: combining risks is less riskier than holding them separately.
\end{itemize}
\vspace{0.3cm}
\item \textcolor{blue}{Example:} let $X$ and $Y$ be i.i.d. r.v.'s which are $\text{Bern}(0.02)$ distributed.
\begin{itemize}
\item Then, $P(X\leq 0) = 0.98$ and $P(Y\leq 0)=0.98$. Thus, $F_X^{-1}(0.975)=F_Y^{-1}(0.975)=0$.
\vspace{0.3cm}
\item For the sum, $X+Y$, we have $P[X+Y=0]=0.98 \cdot 0.98=0.9604$. Thus, $F_{X+Y}^{-1}(0.975)>0$.
\vspace{0.3cm}
\item VaR is not subadditive, since $\text{VaR}(X+Y)$ in this case is larger than $\text{VaR}(X)+\text{VaR}(Y)$.
\end{itemize}
\end{itemize}
\end{frame}

\begin{frame}{Value at Risk IV}
\begin{itemize}
\item Another \textcolor{blue}{drawback of VaR}:
\vspace{0.3cm}
\begin{itemize}
\item[-] it is a single quantile risk measure of a predetermined level $p$;
\vspace{0.3cm}
\item[-] no information about the thickness of the upper tail of the distribution
function from $\text{VaR}_p$ on;
\vspace{0.3cm}
\item[-] whereas stakeholders are interested in both frequency and severity of default.
\end{itemize}
\vspace{0.3cm}
\item Therefore: study other risk measures, e.g. \textcolor{blue}{Tail Value at Risk} (TVaR).
\end{itemize}
\end{frame}

\begin{frame}{Tail Value at Risk}
\begin{itemize}
\item \textcolor{blue}{Definition 3.12:} let $X$ denote a loss r.v., then the Tail Value at Risk of $X$ at the $100p\%$ security level, $\text{TVaR}(p)$, is the \textcolor{blue}{expected loss} \textcolor{green}{given that the loss exceeds the $100p$ percentile} (or: quantile) of the distribution of $X$.
\vspace{0.3cm}
\item We have (assume continuous distribution)
\begin{eqnarray*}
\text{TVaR}_p(X) &=& \textcolor{blue}{E}(X\textcolor{green}{|X>\pi_p}) \\
&=& \frac{\int_{\pi_p}^{\infty} x\cdot \textcolor{green}{f(x)} dx}{\textcolor{green}{1-F(\pi_p)}}.
\end{eqnarray*}
\item We can rewrite this as \quad \textcolor{blue}{[the usual definition of TVaR]}
\begin{eqnarray*}
\text{TVaR}_p(X) &=& \frac{\int_{\pi_p}^{\infty} x dF_X(x)}{1-p} \\
&=& \frac{\int_p^1 \text{VaR}_u(X) du}{1-p},
\end{eqnarray*}
using the substitution $F_X(x) = u$ and thus $x=F_X^{-1}(u)$.
\end{itemize}
\end{frame}

\begin{frame}{Tail Value at Risk}
\begin{itemize}
\item From the definition
\begin{eqnarray*}
\text{TVaR}_p(X) &=& \frac{\int_p^1 \text{VaR}_u(X) du}{1-p},
\end{eqnarray*}
we understand
\vspace{0.3cm}
\begin{itemize}
\item[-] TVaR is the \textcolor{blue}{arithmetic average} of the quantiles of $X$, from level $p$ on;
\vspace{0.3cm}
\item[-] TVaR is `averaging high level VaRs';
\vspace{0.3cm}
\item[-] TVaR \textcolor{blue}{tells us much more about the tail} of the distribution than does VaR alone.
\end{itemize}
\end{itemize}
\end{frame}

\begin{frame}{Tail Value at Risk}
\begin{itemize}
\item Finally, TVaR can also be written as
\begin{eqnarray*}
\text{TVaR}_p(X) &=& E(X|X>\pi_p) \\
&=& \frac{\int_{\pi_p}^{\infty} x f(x)dx}{1-p} \\
&=& \textcolor{blue}{\pi_p} + \frac{\int_{\pi_p}^{\infty} (x-\textcolor{blue}{\pi_p}) f(x) dx}{1-p} \\
&=& \text{VaR}_p(X) + e(\pi_p),
\end{eqnarray*}
with $e(\pi_p)$ the mean excess loss function evaluated at the $100p$th percentile.
\end{itemize}
\end{frame}






\begin{frame}{Tail Value at Risk}
\begin{itemize}
\item We can understand these connections as follows. (Assume continuous r.v.'s)
\vspace{0.3cm}
\item The relation
\begin{eqnarray*}
\text{CTE}_p(X) &=& \text{TVaR}_{F_X(\pi_p)}(X),
\end{eqnarray*}
then follows immediately by combining the other two expressions.
\end{itemize}
\end{frame}

\begin{frame}{Tail Value at Risk}
\begin{itemize}
\item TVaR is a coherent risk measure, see e.g. \href{http://onderwijsaanbod.kuleuven.be/syllabi/e/D0R57BE.htm\#activetab=doelstellingen_idp1406608}{Foundations of Risk Measurement} course.
\vspace{0.3cm}
\item Thus, $\text{TVaR}(X+Y) \leq \text{TVaR}(X)+\text{TVaR}(Y)$.
\vspace{0.3cm}
\item When using this risk measure, we never encounter a situation where combining risks is viewed as being riskier than keeping them separate.
\end{itemize}
\end{frame}


\begin{frame}{Tail Value at Risk}
\begin{itemize}
\item \textcolor{blue}{KPW Example 3.18} \textit{(Tail comparisons)} Consider three loss distributions for an insurance company. Losses
for the next year are estimated to be on average 100 million with standard deviation 223.607
million. You are interested in finding high quantiles of the distribution of losses. Using
the normal, Pareto, and Weibull distributions, obtain the VaR at the 90\%, 99\%, and
99.99\% security levels.
\vspace{0.3cm}
\item \textcolor{blue}{Solution}
\vspace{0.3cm}
\item[] Normal distribution has a lighter tail than the others, and thus smaller quantiles.
\vspace{0.3cm}
\item[] Pareto and Weibull with $\tau<1$ have heavy tails, and thus relatively larger extreme quantiles.
\end{itemize}
\end{frame}

\begin{frame}[fragile]{Tail Value at Risk}
\begin{itemize}
\item \textcolor{blue}{Example 3.18} \textit{(Tail comparisons)} Consider three loss distributions for an insurance company. Losses
for the next year are estimated to be on average 100 million with standard deviation 223.607
million. You are interested in finding high quantiles of the distribution of losses. Using
the normal, Pareto, and Weibull distributions, obtain the VaR at the 99\%, 99.9\%, and
99.99\% security levels.
\vspace{0.3cm}
\begin{verbatim}
> qnorm(c(0.9,0.99,0.999),mu,sigma)
[1] 386.5639 620.1877 790.9976
> qpareto(c(0.9,0.99,0.999),alpha,s)
[1]  226.7830  796.4362 2227.3411
> qweibull(c(0.9,0.99,0.999),tau,theta)
[1]  265.0949 1060.3796 2385.8541
\end{verbatim}
\end{itemize}
\end{frame}

\begin{frame}{Tail Value at Risk}
\begin{itemize}
\item We learn from Example 3.18 that results vary widely depending on the choice of distribution.
\vspace{0.3cm}
\item Thus, the selection of an \textcolor{blue}{appropriate loss model} is highly important.
\vspace{0.3cm}
\item To obtain numerical values of VaR or TVaR:
\vspace{0.3cm}
\begin{itemize}
\item[-] estimate from the data directly;
\vspace{0.3cm}
\item[-] or use distributional formulas, and plug in parameter estimates.
\end{itemize}
\end{itemize}
\end{frame}

\begin{frame}{Tail Value at Risk}
\begin{itemize}
\item When estimating VaR directly from the data:
\vspace{0.3cm}
\begin{itemize}
\item[-] use R to get quantile from the empirical distribution;
\vspace{0.3cm}
\item[-] R has 9 ways to estimate a VaR at level $p$ from a sample of size $n$, differing in the way the interpolation between order statistics close to $np$ .
\end{itemize}
\vspace{0.3cm}
\item When estimating TVaR directly from the data:
\vspace{0.3cm}
\begin{itemize}
\item[-] take average of all observations that exceed the threshold (i.e.\ $\pi_p$);
\end{itemize}
\vspace{0.3cm}
\item \textcolor{blue}{Caution:} we need a large number of observations (and a large number of observations $> \pi_p$) in order to get reliable estimates.
\vspace{0.3cm}
\item When not may observations in excess of the threshold are available:
\vspace{0.3cm}
\begin{itemize}
\item[-] construct a loss model;
\vspace{0.3cm}
\item[-] calculate values of VaR and TVaR directly from the fitted distribution.
\end{itemize}
\end{itemize}
\end{frame}

\begin{frame}{Tail Value at Risk}
\begin{itemize}
\item For example
\begin{eqnarray*}
&&\text{TVaR}_p(X) = E(X|X>\pi_p) \\
&=& \pi_p + \frac{\textcolor{blue}{\int_{\pi_p}^{\infty}} (x-\pi_p) f(x) dx}{1-p} \\
&=& \pi_p + \frac{\textcolor{blue}{\int_{-\infty}^{\infty}} (x-\pi_p) f(x) dx \textcolor{blue}{-\int_{-\infty}^{\pi_p}} (x-\pi_p) f(x) dx }{1-p} \\
&=& \pi_p + \frac{E(X)-\int_{-\infty}^{\pi_p} xf(x) dx -\pi_p (1-F(\pi_p))}{1-p} \\
&=& \pi_p + \frac{E(X) - E[\min{(X,\pi_p)}]}{1-p} = \pi_p + \frac{E(X)-E(X \wedge \pi_p)}{1-p},
\end{eqnarray*}
see Appendix A for those expressions.
\end{itemize}
\end{frame}

\section{Reinsurance}

\begin{frame}%[shrink=2]
\frametitle{Reinsurance}
\begin{itemize}
\item A major difference between reinsurance and primary insurance is that a reinsurance program is generally tailored more closely to the buyer
\item There are two major types of reinsurance
\begin{itemize}
\item Proportional
\item Excess of Loss
\end{itemize}
 \item A proportional treaty is an agreement between a reinsurer and a ceding company (the reinsured) in which the reinsurer assumes a given percent of losses and premium.
\end{itemize}
\end{frame}

\begin{frame}%[shrink=2]
\frametitle{Proportional Reinsurance}
\begin{itemize}
 \item A proportional treaty is an agreement between a reinsurer and a ceding company (the reinsured) in which the reinsurer assumes a given percent of losses and premium.
\item The simplest example of a proportional treaty is called \textit{Quota Share}.
\begin{itemize}
 \item In a quota share treaty, the reinsurer receives a flat percent, say 50\%, of the premium for the book of business reinsured.
 \item In exchange, the reinsurer pays 50\% of losses, including allocated loss adjustment expenses
 \item The reinsurer also pays the ceding company a ceding commission which is designed to reflect the differences in underwriting expenses incurred.
\end{itemize}
\item The amounts paid by the direct insurer and the reinsurer are defined as follows:
\end{itemize}
$$ Y_{insurer} = c X \ \ \ \ \ Y_{reinsurer} = (1-c) X$$
Note that $Y_{insurer}+Y_{reinsurer}=X$.
\end{frame}

\begin{frame}%[shrink=2]
\frametitle{Surplus Share Proportional Treaty}
\begin{itemize}
\item Another proportional treaty is known as \textit{Surplus Share}; these are common in property business.
\item A surplus share treaty allows the reinsured to limit its exposure on any one risk to a given amount (the \textit{retained line}).
\item The reinsurer assumes a part of the risk in proportion to the amount that the insured value exceeds the retained line, up to a given limit (expressed as a multiple of the retained line, or ``number'' of lines).
\item For example, let the retained Line be \$100,000 and let the given limit be 4 lines (\$400,000). Then, if $X$ is the loss, the reinsurer's portion is
$\min(400000, (X-100000)_+)$.
\end{itemize}
\end{frame}


\begin{frame}%[shrink=2]
\frametitle{Excess of Loss Reinsurance}
\begin{itemize}
\item Under this arrangement, the direct insurer sets a retention level $M (>0)$ and pays in full any claim for which $X  \le M$.
\item The direct insurer retains an amount $M$ of the risk.
\item For claims for which $X > M$, the direct insurer pays $M$ and the reinsurer pays the remaining amount $X-M$.
\item The amounts paid by the direct insurer and the reinsurer are defined as follows:
\end{itemize}
$$
Y_{insurer} =
\begin{cases}
X & X \le M\\
M & X >M \\
\end{cases} \ \ \ \ = \min(X,M) = X \wedge M
$$

$$
Y_{reinsurer} =
\begin{cases}
0 & X \le M\\
X- M & X >M \\
\end{cases} \ \ \ \  = \max(0,X-M)
$$
Note that $Y_{insurer}+Y_{reinsurer}=X$.
\end{frame}

\begin{frame}%[shrink=2]
\frametitle{Relations with Personal Insurance}
\begin{itemize}
\item We have already seen the needed tools to handle reinsurance in the context of personal insurance
\begin{itemize}
\item For a proportional reinsurance, the transformation  $Y_{insurer} = c X $ is the same as a ``coinsurance'' adjustment in personal insurance
\item For excess of loss reinsurance, the transformation $Y_{reinsurer} = \max(0,X-M)$ is the same as an insurer's payment with a deductible $M$ and $Y_{insurer} = \min(X,M) = X \wedge M$ is equivalent to what a policyholder pays with deductible $M$.
\end{itemize}
\item Reinsurance applications suggest introducing \textit{layers of coverage}, a (small) mathematical extension.
\end{itemize}
\end{frame}

\begin{frame}%[shrink=2]
\frametitle{Layers of Coverage}
\begin{itemize}
\item Instead of simply an insurer and reinsurer or an insurer and a policyholder, think about the situation with all three parties, a policyholder, insurer, and reinsurer, who agree on how to share a risk.
\item In general, we consider $k$ parties. If $k=4$, it could be an insurer and three different reinsurers.
\item Consider a simple example:
\begin{itemize}
\item Suppose that there are $k=3$ parties. The first party is responsible for the first 100 of claims, the second responsible for claims from 100 to 3000, and the third responsible for claims above 3000.
\item If there are four claims in the amounts 50, 600, 1800 and 4000, they would be allocated to the parties as follows:
\end{itemize}
\end{itemize}
\begin{table}[H]\scalefont{0.8}
 \centering
    \begin{tabular}{l|cccc|l}
    \hline
 Layer          & Claim 1&  Claim 2 &  Claim 3 &  Claim 4 & Total \\ \hline
(0, 100]        & 50     &      100 &      100 &      100 &  350 \\
(100, 3000]     & 0      &      500 &     1700 &     2900 & 5100 \\
(3000, $\infty$)& 0      &      0   &        0 &     1000 & 1000 \\ \hline
Total & 50 & 600 & 1800 & 4000 & 6450 \\ \hline
    \end{tabular}
 \end{table}
\end{frame}

\begin{frame}%[shrink=2]
\frametitle{Layers of Coverage II}
\begin{itemize}\scalefont{0.9}
\item Mathematically, partition the positive real line into $k$ intervals using the cut-points $0 = c_0 < c_1 < \cdots < c_{k-1} < c_k = \infty$.
\begin{itemize}\item The $j$th interval is $(c_{j-1}, c_j]$.\end{itemize}
\item Let $Y_j$ be the amount of risk shared by the $j$th party
\item To illustrate, if a loss $x$ is such that $c_{j-1} <x \le c_j$, then
$$
\left(\begin{array}{c}
Y_1\\ Y_2 \\ \vdots \\ Y_j \\Y_{j+1} \\ \vdots \\Y_k
\end{array}\right)
=\left(\begin{array}{c}
c_1-c_0 \\ c_2-c_1  \\ \vdots \\ x-c_{j-1}  \\ 0 \\ \vdots \\0
\end{array}\right)
$$
\item More succinctly, we can write
$$Y_j = \min(X,c_j) - \min(X,c_{j-1}) $$
\end{itemize}
\end{frame}

\begin{frame}%[shrink=2]
\frametitle{Layers of Coverage III}
\begin{itemize}
\item With the expression $Y_j = \min(X,c_j) - \min(X,c_{j-1}) $, we see that the $j$th party is responsible for claims in the interval $(c_{j-1}, c_j]$
\item Note that $X = Y_1 + Y_2 + \cdots + Y_k$
\item The parties need not be different.
\begin{itemize}
\item For example, suppose that a policyholder is responsible for the first 500 of claims and all claims in excess of 100,000. The insurer takes claims between 100 and 100,000.
\item Then, we would use $c_1 = 100$, $c_2 =100000$.
\item The policyholder is responsible for $Y_1 =\min(X,100)$ and $Y_3 = X - \min(X,100000) = \max(0, X-100000)$.
\end{itemize}
\item See the Wisconsin Property Fund site for more info on layers of reinsurance,
\url{https://sites.google.com/a/wisc.edu/local-government-property-insurance-fund/home/reinsurance}
\end{itemize}
\end{frame}

\end{document}





\begin{frame}[shrink=2]
\frametitle{xxx}
\begin{itemize}
\item
\end{itemize}
\end{frame}


\end{document}

\textcolor{blue}{temp}

